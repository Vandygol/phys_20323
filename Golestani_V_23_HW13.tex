
%%%%%%%%%%%%%%%%%%%%%%%%%%%%%%%%%%%%%%%%%%%%%%%%%%%%%%%%%%%%
\documentclass[12pt]{article}
\usepackage{amsmath}
\usepackage{epsfig}
\usepackage{times}
\usepackage{color}
\usepackage{xcolor}
\renewcommand{\topfraction}{1.0}
\renewcommand{\bottomfraction}{1.0}
\renewcommand{\textfraction}{0.0}
\setlength {\textwidth}{6.6in}
\hoffset=-1.0in
\oddsidemargin=1.00in
\marginparsep=0.0in
\marginparwidth=0.0in                                                                               
\setlength {\textheight}{9.0in}
\voffset=-1.00in
\topmargin=1.0in
\headheight=0.0in
\headsep=0.00in
\footskip=0.50in                                         
\setcounter{page}{1}
\begin{document}
\def\pos{\medskip\quad}
\def\subpos{\smallskip \qquad}
\newfont{\nice}{cmr12 scaled 1250}
\newfont{\name}{cmr12 scaled 1080}
\newfont{\swell}{cmbx12 scaled 800}
%%%%%%%%%%%%%%%%%%%%%%%%%%%%%%%%%%%%%%%%%%%%%%%%%%%%%%%%%%%%

\begin{center}
{\LARGE
PHYS 20323/60323: Fall 2023 - LaTeX Example
}\\
\end{center}
%%%%%%%%%%%%%%%%%%%%%%%%%%%%%%%%%%%%%%%%%%%%%%%%%%%%%%%%%%%%
\noindent {\bf 1. The following questions refer to stars in the Table below.} \

Note: There may be multiple answers. \

\begin{center}
\begin{tabular}{|c|c|c|c|c|c|}
\hline
Name & Mass & Luminosity & Lifetime & Temperature & Radius \\
\hline
$\eta$ Car.   & 60. \textsl{$M_\odot$}   &  $10^6$ \textsl{$L_\odot$}  & $8.0\times10^5$ years & &  \\
\hline
$\epsilon$ Eri.   & 6.0 \textsl{$M_\odot$}   &  $10^3$ \textsl{$L_\odot$}  &  & 20,000 K &  \\
\hline
$\delta$ Scu.   & 2.0 \textsl{$M_\odot$}   &  & $5.0\times10^8$ years & & 2 \textsl{$R_\odot$}  \\
\hline
$\beta$ Cyg.   & 1.3 \textsl{$M_\odot$}   &  3.5 \textsl{$L_\odot$}  &  & &  \\
\hline
$\alpha$ Cen.   & 1.0 \textsl{$M_\odot$}   &  &  & & 1 \textsl{$R_\odot$} \\
\hline
$\gamma$ Del.   & 0.7 \textsl{$M_\odot$}   &   & $4.5\times10^{10}$ years & 5,000 K &  \\
\hline
\end{tabular}\vskip 0.2in
\end{center}


(a) (4 points) Which of these stars will produce a planetary nebula.\\


(b) (4 points) Elements heavier than \textit{Carbon} will be produced in which stars \\

%%%%%%%%%%%%%%%%%%%%%%%%%%%%%%%%%%%%%%%%%%%%%%%%%%%%%%%%%%%%
\noindent {2. An electron is found to be in the spin state (in the z-basis): $\chi= A 
    \begin{pmatrix}
        3\textit{i} \\ 4
    \end{pmatrix}$} \\

(a) (5 points) Determine the possible values of A such that the state is normalized.\\


(b) (5 points) Find the expectation values of the operators ${\color{red}S_x}, {\color{violet}S_y}, {\color{orange}S_z}, \Vec{S^2}.$ \\

\noindent{        The matrix representations in the \textsl{z}-basis for the components of electron spin operators are given by:} \\

\noindent{\Large${{\color{red}S_x=  \frac{\hbar}{2}\begin{pmatrix}
        0 & 1\\ 1 & 0 \end{pmatrix} ; \\} \hspace{1cm} \color{violet}S_y=  \frac{\hbar}{2}\begin{pmatrix} 0 & \textsl{-i}\\ \textsl{i} & 0 \end{pmatrix};\hspace{1cm}{\color{orange}S_x=  \frac{\hbar}{2}\begin{pmatrix}1 & 0\\ 0 & -1 \end{pmatrix} ; \\} }$} \\


\noindent{3. The average electrostatic field in the earth’s atmosphere in fair weather is approximately given:} \\

\begin{equation}
    \large{\Vec{E}=E_0(Ae^{-\alpha\textit{z}}+Be^{-\beta\textit{z}})\hat{\mathit{z,}}}
\end{equation} \\

Where $A$, $B$, $\alpha$, $\beta$ are positive constants and $\textit{z}$ is the height above the (locally flat) earth surface.\\


(a) (5 points) Find the average charge density in the atmosphere as a function of height \\

(b) (5 points) Find the electric potential as a function height above the earth. \\
\vspace{20pt} \\
\noindent{Latex Example} \hspace{5.4in} 31

\end{document}
